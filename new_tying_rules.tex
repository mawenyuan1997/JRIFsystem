
\documentclass[12pt]{article}
\usepackage[T1]{fontenc}
\usepackage{semantic}

\begin{document}
We assume there is a restrictiveness relation $\sqsubseteq$ that compares labels in terms
of the restrictions they impose. We write $l\sqsubseteq l'$ to denote that $l'$ is at least as
restrictive as $l$. In our case, $l'$ is a subset of $l$.

Restrictiveness relation $\sqsubseteq$ essentially define allowed flows between labels. If $l\sqsubseteq l'$
, then values in variables tagged with $l$ are allowed to flow to variables
tagged with $l'$.
\[
\inference{\Gamma\vdash x:l_x, ctx\sqcup l_x\sqcup a^$p^*$\sqsubseteq \Gamma(y), l_x\sqcup a^$p^*$\neq 0}{\Gamma, ctx\vdash y:=reclassify(x, a)}
\]
$ap^*$ represents all KAT strings starting with atom $a$ (with any actions after).

Explanation: $l_x\sqsubseteq ctx\sqcup l_x\sqcup ap^{*}\sqsubseteq \Gamma(y)$, $ap^{*}\sqsubseteq ctx\sqcup l_x\sqcup ap^{*}\sqsubseteq \Gamma(y)$, $\Gamma(y)$ is the subset of both $l_x$ and $ap*$, so it is equivalent to say $\Gamma(y)$ is all KAT strings in $l_x$ with $a$ as its head.
$$
    \delta_p(E)&=\sum_{\alpha} \delta_{\alpha p}(E)
$$
\[
\inference{\Gamma\vdash y:l_y, ctx\sqcup \delta_f(l_y)\sqsubseteq \Gamma(z), \delta_f(l_y)\neq 0}{\Gamma, ctx\vdash z:=reclassify(y, f)}
\]

Explanation: since $l_y$ already starts with a certain $\alpha$, summing across all atoms will be the same as $\alpha f$

\end{document}
